\documentclass[uplatex, a4paper]{jsarticle}
\usepackage{amsmath, amssymb}
\usepackage{amsthm, newtxtext, newtxmath}
\usepackage{thmtools}
\usepackage{thmmacros}

\title{リソース理論のノート}
\author{澤田 太郎}

\begin{document}
\maketitle
これはリソース理論の勉強ノートです. 主にメモです.

\newpage
\section{記号等の前提知識}
この文書では以下の宣言を断りなく用います.
\begin{itemize}
  \item 「量子系」と言ったときは、observableの集合、すなわち、あるHilbert空間上の有界作用素のなす$C^*$代数の部分$C^*$代数を指します.
  \item 量子系A,Bに対し、$\mathrm{CPTP}(A,B)$をAからBへのCPTP写像全体の集合とします.
  \item また、量子系A, Bに対し、$\mathfrak{F}(A,B)$を$\mathrm{CPTP}(A,B)$のsubsetとします.
\end{itemize}

\newpage
\section{種々の定義}
\begin{mtvn}
  リソース理論で主題にしたいのは以下の要素です.
  \begin{itemize}
    \item 注目系が「自由」なときに可能なこと
    \item 逆に、「自由」なときに不可能なこと
    \item さらに、注目系に「リソース」がくっついたときに初めて可能になること
  \end{itemize}
  この「自由」「リソース」という単語は、文脈に応じて柔軟に意味を変える(定義が変わる)概念を抽象化した(モチベーションだけを一人歩きさせた)ものです.
  そのため、ここではきわめてぼやけた運用をしておきます.
\end{mtvn}
\begin{defn}
  「自由な写像の宣言」$\mathfrak{F} := \{ \mathfrak{F}(A,B)| \:A,B:\text{量子系} \}$が\underline{リソース理論}であるとは、以下の(R1)から(R6)をみたすことをいう. 
  また、任意の$\mathfrak{F}(A) := \mathfrak{F}(\mathbb{C}, A)$の元を「自由な状態」という.
  \begin{align*}
    \textrm{(R1)}& \quad \forall A: \mathrm{id}^A \in \mathfrak{F}(A, A) \\
    \textrm{(R2)}& \quad \forall A, \forall B, \forall C: \left( \mathscr{E} \in \mathfrak{F}(A, B), \:\mathscr{N} \in \mathfrak{F}(B, C) 
                  \Rightarrow \mathscr{E} \circ \mathscr{N} \in \mathfrak{F}(A, C) \right) \\
    \textrm{(R3)}& \quad \mathfrak{F}(A, \mathbb{C}) = \mathrm{CPTP}(A, \mathbb{C}) = \{ \mathrm{Tr} \} \\
    \textrm{(R4)}& \: \textrm{(完全自由性)} \\
                &\quad \forall A, \forall B, \forall C: \forall \mathscr{E} \in \mathfrak{F}(A,B): 
                  \mathscr{E}\otimes \mathrm{id}^C \in \mathfrak{F}(A\otimes C, B\otimes C) \\
    \textrm{(R5)}& \quad \forall A: \mathfrak{F}(A) \: \mathrm{is \: closed.} \\
    \textrm{(R6)}& \quad \forall A: \mathfrak{F}(A) \: \mathrm{is \: convex.}
  \end{align*}
\end{defn}
\begin{prop}["黄金律"]
  \[ \rho \in \mathfrak{F}(A), \mathscr{E} \in \mathfrak{F}(A, B) 
  \Rightarrow \mathscr{E}(\rho) \in \mathfrak{F}(B) \]
  これは自由な状態に自由な操作を施すと、出力も自由であることを示している.(証明はby def.)
\end{prop}
\begin{rmk}
  どんな自由状態も他の自由状態に自由演算で移り変われる.
\end{rmk}

\end{document}